\documentclass{article}

\usepackage{parskip}
\usepackage{titlesec}
\usepackage{hyperref}
\usepackage{xstring}
\usepackage{graphicx}

% Move the title up a bit.
\usepackage{titling}
\setlength{\droptitle}{-10ex}

% Describe how code should be syntax-highlighted.
\usepackage{minted}
\usemintedstyle{borland}

% Don't use curly quotes in code listings.
\usepackage{upquote}
\usepackage{textcomp}
\AtBeginDocument{%
\def\PYZsq{\textquotesingle}%
}

% Use the "courier" font for code listings.  Unlike the default monospace font, 
% courier supports bold and doesn't look funny when italicized.
\usepackage{courier}

% Describe how the TOC should be formatted.
\usepackage{tocloft}
\renewcommand\cftsecleader{\cftdotfill{\cftdotsep}}
\renewcommand\cftsecfont{\normalfont}
\renewcommand\cftsecpagefont{\normalfont}
\setlength{\cftbeforesecskip}{0.0ex}

% Specify how hyperlinks should be colored.
\hypersetup{
  colorlinks   = true,
  urlcolor     = blue,
  linkcolor    = black,
  citecolor    = black
}

% In description lists, start each description on it's own line.
\usepackage{enumitem}
\setlist[description]{style=nextline}

% Don't number sections, and start each section on a new page.
\setcounter{secnumdepth}{0}
\newcommand{\sectionbreak}{\clearpage}

\makeatletter
\newcommand\organizedtoc{%
    \subsubsection*{Making the code available}\@starttoc{use}%
    \subsubsection*{Documenting the code}\@starttoc{doc}%
    \subsubsection*{Testing the code}\@starttoc{test}}
\newcommand\organizedsection[2]{
 \section{#2}
 \addcontentsline{#1}{section}{#2}
 \setcounter{footnote}{1}}
\makeatother

% Define a command to typeset links in a way that's readable in print.
\usepackage{xifthen}
\newcommand{\link}[2]{#1\footnote{\url{#2}}}

% Define a command for typesetting module names.  A URL can optionally be 
% given, in which case it will be added as a footnote and a hyperlink.
\usepackage{xifthen}
\newcommand{\module}[2][]{%
 \ifthenelse{\isempty{#1}}%
 {\texttt{#2}}%
 {\link{\texttt{#2}}{#1}}%
}

% Use symbols to mark footnotes.
\renewcommand*{\thefootnote}{\fnsymbol{footnote}}

% Define a command to include files from the kbkdna module.
\newcommand{\kbkdnafile}[2]{%
\begin{listing}[h]%
 \inputminted[fontsize=\small]{#1}{../kbkdna/#2}%
 \caption{\texttt{#2}}%
\end{listing}}

\begin{document}

\title{How to publish python code}
\author{Kale Kundert}
\date{2016 Kortemme Lab Retreat}
\maketitle{}

Once you get some code to work, it's easy to not bother making that code easy 
for other people to use.  Usually you have more pressing things to worry about, 
and it's hard to even know what you could do to help other people use your code 
anyway.  However, especially as scientists, we have a responsibility to make 
sure the code we write easily is accessible to anyone who wants or needs to use 
it.

In this tutorial, we'll walk through all the steps required to publish some 
python code.  The example code we'll be working with will manipulate DNA 
sequences in simple ways.  Specifically, it'll calculate lengths, GC contents, 
and reverse complements.  We'll expose this functionality both as a library and 
as a command-line tool.

Broadly speaking, there are three things we need to do to publish code.  These 
are making the code available, documenting the code, and testing the code.  
Each of these larger objectives can be broken down into a handful of very 
concrete steps which this tutorial will describe:

\organizedtoc

\organizedsection{use}{Organizing your files}

Most python projects organize their files in the same way.  This standard 
directory layout is pretty simple and works well in every situation I've ever 
encountered.  By adhering to it, you make your code easier for others to grok 
and you avoid getting tied into knots by bad organizational decisions.

The first step is to choose a name for your project.  This name has to be 
unique (i.e. there can't be any other python packages with the same name) and 
it should somehow reflect what your code actually does.  The name I'll use for 
this tutorial is \module{kbkdna}, because my initials are \texttt{kbk} and the 
code will deal with DNA.  You should use your own initials instead, because you 
won't be able to publish a package with the same name as mine.

\begin{listing}[h]
 \begin{minted}[fontsize=\small]{bash}
 # Make a directory for the project and move into it.
 $ mkdir kbkdna
 $ cd kbkdna

 # Make a README file.
 $ touch README.rst

 # Make a LICENSE file.
 $ touch LICENSE.txt

 # Make a directory for your python code.
 $ mkdir kbkdna
 $ touch kbkdna/__init__.py

 # Make a directory for your documentation.
 $ mkdir docs

 # Make a directory for your tests.
 $ mkdir tests

 # Make sure everything was created properly.
 $ ls
 docs/  kbkdna/  README.rst  LICENSE.txt
\end{minted}
\end{listing}

\organizedsection{misc}{Writing some example functions}

Below are the four functions making up the library we want to publish.  Nothing 
about this code is particular to how we will publish it; it's just regular 
python code.  Copy it into \texttt{kbkdna/dna.py}:

\kbkdnafile{python}{kbkdna/dna.py}

\organizedsection{doc}{Providing command-line help with \module{docopt}}

Along with our library, we want to provide a program to manipulate DNA 
sequences from the command-line.  This program will read a sequence and a 
desired manipulation (i.e. length, GC content, reverse complement) from the 
command line, and will print out the result of that manipulation.  Documenting 
the arguments that this program expects is a crucial part of making this 
program usable for other people.  This should be done in two ways:

\begin{enumerate}
 \item If the program receives arguments that don't make sense, it should print 
  a brief usage hint.
 \item If the user provides the \texttt{-h} or \texttt{--help} flag, the 
  program should print a detailed help message.  This message should include a 
  few sentences saying what the program is supposed to do and a description of 
  every single argument the program takes.
\end{enumerate}

There are a number of ways to do both these things in python, but the 
\module[http://docopt.org/]{docopt} module is by far my favorite.  We just need 
to write a help message at the top of our script describing the arguments we 
expect.  \module{docopt} will read that message and use it to parse the 
command-line arguments for us.  If there are any problems, it will print a 
usage hint and end the program.  If not, it will give us a dictionary 
containing the parsed arguments.

The thing I love about \texttt{docopt} is that it combines a good help message, 
which is good for users, with easy command-line argument parsing, which is good 
for me.  I've found that I'm much more likely to write documentation when 
there's an immediate benefit in it for me!

There's an example on the following page.  Copy it into \texttt{kbkdna/cli.py}.  
There are parts of this code that are particular to how we will publish it, 
specifically the relative import and the main function that's never called, so 
don't bother trying to run it yet.

\kbkdnafile{python}{kbkdna/cli.py}

\organizedsection{doc}{Writing a \texttt{README.rst} file}

A \texttt{README} file should briefly explain what the project is supposed to 
do, how to install it, and how to use it.  I also like to include a few 
``badges'' to advertise that the project can be installed using the standard 
tools, that it passes its tests, and that its documentation is available online 
(which are all things we'll get to in this tutorial).

The \texttt{*.rst} suffix means that the file is reStructuredText.  This is a 
simple file format is supposed to look good as raw text while still being easy 
to convert to HTML.  The two important things to know are that you can start a 
new section by underlining the title of that section with equal signs or 
dashes, and you enter a block of code by ending the previous paragraph with two 
colons and indenting the code block.

\kbkdnafile{text}{README.rst}

\organizedsection{use}{Choosing a license}

Choosing a license is important because it lets other people know how they can 
use your code.  The website \url{choosealicense.com} is a good resource where 
you can compare a number common licenses and find one you like.  The two most 
common choices for academic code are the MIT license and the GPLv3 license:

\begin{description}
\item[MIT License] No restrictions on how your code can be used.  This allows 
 your code to be included with commercial software.
\item[GPLv3 License] Requires that any derivatives of your code be made 
 publicly available.  This effectively forbids your code from being included 
 with commercial software.
\end{description}

I'll use the MIT license for \module{kbkdna} because it's short enough to fit 
on one page.

\kbkdnafile{text}{LICENSE.txt}

\organizedsection{use}{Writing \texttt{setup.py} and uploading to PyPI}

The best way to install python software that other people have written is to 
use a command called \module[https://packaging.python.org/installing/]{pip}.  
You just need to give the name of the software you want to install.  For 
example, to install \module{docopt}:

\begin{minted}{bash}
 $ pip install docopt
\end{minted}

In order for other people to be able to install our software using 
\module{pip}, we need to upload it to the \link{Python Package Index 
(PyPI)}{https://pypi.python.org/pypi}, a global repository of python software 
that anyone can contribute to.  The first step in this process is to specify 
some basic information about our project in a file called \texttt{setup.py}.

\kbkdnafile{python}{setup.py}

The \texttt{name} field is the name \module{pip} will use to refer to our 
project, e.g. \texttt{pip install kbkdna}.  The \texttt{packages} field is a 
list of directories containing python code that will be installed.  The 
\texttt{install\_requires} field is a list of other packages on PyPI that our 
project makes use of.  Whenever \module{pip} installs our project, it will 
automatically install these as well.  The rest of the fields are just optional 
metadata and are hopefully pretty self-explanatory.

The next step is to register an account on PyPI:

\begin{figure}[h]
 \frame{\includegraphics[width=\textwidth]{screenshots/pypi}}
 \caption{\url{https://pypi.python.org/pypi}}
\end{figure}

Once you've done that, the third an final step is to run the following two 
commands to register and upload your project.  You only need to run the 
\texttt{register} command once for each of your projects.  You can run the 
\texttt{sdist upload} command any time you want to upload a new version of your 
project, but you need to increment the version number in \texttt{setup.py} each 
time.

\begin{minted}{bash}
 $ python setup.py register
 $ python setup.py sdist upload
\end{minted}

\organizedsection{doc}{Writing documentation with Sphinx}

Sphinx is a tool that 

\begin{minted}{bash}
 $ pip install sphinx
 $ sphinx-quickstart docs
\end{minted}

\organizedsection{doc}{Uploading documentation to ReadTheDocs}

\organizedsection{test}{Writing tests with \module{py.test}}

\organizedsection{test}{Running tests on TravisCI}

\organizedsection{misc}{Using \module{cookiecutter}}

\end{document}

